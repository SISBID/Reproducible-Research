\documentclass[12pt,t]{beamer}
\usepackage{graphicx}
\setbeameroption{hide notes}
\setbeamertemplate{note page}[plain]
\usepackage{listings}

\input{LaTeX/header.tex}

%%%%%%%%%%%%%%%%%%%%%%%%%%%%%%%%%%%%%%%%%%%%%%%%%%%%%%%%%%%%%%%%%%%%%%
% end of header
%%%%%%%%%%%%%%%%%%%%%%%%%%%%%%%%%%%%%%%%%%%%%%%%%%%%%%%%%%%%%%%%%%%%%%

% title info
\title{12: Summary + Extras}
\author{}
\date{\href{https://bit.ly/SISBID3}{\tt \color{foreground} bit.ly/SISBID3}}

\begin{document}

% title slide
{
\setbeamertemplate{footline}{} % no page number here
\frame{
  \titlepage
  \note{We'll briefly summarize all that we discussed, and further
    touch on how to share your research data and code. Then we'll give
    pointers to the many things that we didn't have time to talk about.
} }


\begin{frame}[c]{}

\begin{center}
\large
The most important tool is the {\hilit mindset},\\
when starting, that the end product \\
will be reproducible.
\end{center}

\hfill
{\lolit
{\textendash} \href{http://odin.mdacc.tmc.edu/~kabaggerly/}{Keith Baggerly}
}

\note{So true. Desire for reproducibility is step one.
}
\end{frame}




\begin{frame}[c]{Steps toward reproducible research}


  \bi
\item Slow down
\item Organize; document
\item Everything with code
\item Scripts → RMarkdown
\item Code → functions → packages
\item Version control with Git
\item Automation with Make
\item Choose a license
\item Share your work with others
  \ei


\note{Moving from ``standard practice'' to ``fully reproducible'' is
  hard. There are a lot of tools to learn and a lot of workflow
  changes to make. Don't try to change everything all at once. Focus
  on improving one aspect at a time, ideally jointly with your friends
  and colleagues. Your goal should be to have each project be a bit
  better organized than the previous.
}
\end{frame}



\begin{frame}[c]{Challenges}


  \bi
\item Daily maintenance
  \bi
\item READMEs up to date?
\item Documentation match code?
  \ei
\item Cleaning up the junk
  \bi
\item Move defunct stuff into an {\tt Old/} subdirectory?
  \ei
\item Start over from the beginning, nicely?
  \ei


\note{The organization of a project is dependent on your worst day
  with it. You keep everything carefully arranged for months and then
  one day you need to rush, rush, rush to get a manuscript out the
  door, and you leave a big mess.

  And a common problem is that you don't really know what you're doing
  until it's all done. So if you can, spend a week at the end making a
  separate, clean version of the whole thing.
}
\end{frame}



\begin{frame}[c]{Unmentioned}

  \bi
\item Code review / paired programming
\item Software testing (and debugging)
\item Capturing versions of dependent software (e.g.
  \href{https://rstudio.github.io/packrat}{packrat})
  \ei

\end{frame}

\end{document}
